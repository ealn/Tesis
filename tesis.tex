%% ##############################################################################################
%% Plantilla estándar (en el caso de que exista algo parecido a un estándar) para el documento de tesis para la maestría en ciencias computacionales del INAOEP.
%% Fecha última modificación: 23 de agosto de 2010
%% Editada por: R. Omar Chávez G.
%% Nota: Cualquier mejora o modificación favor de comentarla y distribuirla.
%% ----------------------------------------------------------------------------------------------
%% Registro de modificaciones
%%
%% ##############################################################################################

\documentclass[12pt,letterpaper,spanish]{book}
%\spanishdecimal{.}
% para emcabezados y pies de página
\usepackage{fancyhdr}

% para usar de manera natural los acentos, tildes y así -- además del respectivo diccionario de español
%\usepackage[utf8]{inputenc}
\usepackage[latin1]{inputenc}
\usepackage[spanish,activeacute]{babel}
\spanishdecimal{.}
\usepackage{amsmath}
\usepackage{amssymb}
\usepackage{amsthm}
\usepackage{algorithmic}
\usepackage{algorithm}
%\usepackage{amsmath}
%\usepackage{amssymb}
\usepackage{cool} %pa sumatorias cool

% para tablas que se pasan de lanza, digo, de largas
\usepackage{longtable}
\usepackage{array}

% para incluir los apéndices
\usepackage{appendix}

% para incluir gráficas y subgráficas dentro de las gráficas etc., etc.
\usepackage{graphicx}
\usepackage{subfigure}
\usepackage{wrapfig}
\usepackage{float}
\usepackage{caption}
%\usepackage{subcaption}

%\usepackage{subfig}%%%%%
%\usepackage{float}   %%%%% tambien nuevo
%\usepackage[hKV-listi]{subfig} %%%% nuevo

% para definir algunos márgenes personalizados
\usepackage{anysize}
\marginsize{1.5in}{1in}{1in}{1in}
\usepackage[nottoc]{tocbibind}
\usepackage{setspace}

% para tablas con celdas compartidas
\usepackage{multirow}
\usepackage{fancybox}
\usepackage{bigstrut}
\usepackage{booktabs}


% para incluir colores en texto o en celdas de las tablas
\usepackage{color}
\usepackage{colortbl}

% para el estilo natbib de bibliografías (sino lo usan no va)
\usepackage{cite}

\usepackage{titlesec}

% para incluir páginas de un archivo pdf
\usepackage{pdfpages}

%##############################################################################
% para renombrar algunos títulos
\renewcommand{\appendixname}{Apéndices}
\renewcommand{\appendixtocname}{Apéndices}
\renewcommand{\appendixpagename}{Apéndices}

%##############################################################################
% incluimos información personalizada para algunos valores de la plantilla
\newcommand{\tstamp}{\small [\thepage]}
\newcommand{\Escuela}{\scshape \scriptsize}
\newcommand{\Uni}{\scshape \scriptsize Universidad Aut�noma de Guadalajara}
\newcommand{\TEMA}{\scshape \scriptsize ``T�tulo de la Tesis''}


%##############################################################################
% redefinimos los espacios para las figuras

\newcommand{\figura}[5]{
  \begin{figure}[#5]
  \begin{center}
  \includegraphics[#2]{#1}
  \caption[#3]{#3}
  \label{#4}
  \end{center}
  \end{figure}
}
%##############################################################################
% redefinimos los espacios para las tablas
\newenvironment{tabla}[2]{
  \begin{table}[#2]
  \begin{center}
  \caption[#1]{#1}
}{
  \end{center}
  \end{table}
}

%##############################################################################
% renombramos algunas constantes para entornos matemáticos
\def\sen{{\rm sen \ }}
\def\cos{{\rm cos \ }}

\def\a{\mathbf a}
\def\b{\mathbf b}
\def\c{\mathbf c}
\def\d{\mathbf d}
\def\e{\mathbf e}
\def\ex{\mathrm e}
\def\f{\mathbf f}
\def\g{\mathbf g}
\def\h{\mathbf h}
\def\k{\mathbf k}
\def\n{\mathbf n}
\def\m{\mathbf m}
\def\p{\mathbf p}
\def\q{\mathbf q}
\def\P{\mathbf P}
\def\R{\mathbf R}
\def\T{\mathbf T}
\def\L{\mathbf L}
\def\r{\mathbf r}
\def\u{\mathbf u}
\def\v{\mathbf v}
\def\w{\mathbf w}
\def\x{\mathbf x}
\def\y{\mathbf y}
\def\z{\mathbf z}
\def\sen{{\rm sen \ }}

%################################################################################
% rebombramos las etiquetas para retuilizar la información antes ingresada
\renewcommand{\labelenumi}{\textbf{\theenumi}.-}
\renewcommand{\labelenumii}{\textbf{\theenumi}.-}
\renewcommand{\labelenumiii}{\textbf{\theenumi}.-}
\renewcommand{\labelenumiv}{\textbf{\theenumi}.-}



%#################################################################################
% especificamos casos extraodinarios para la división de palabras
\hyphenation{es-pe-cia-li-dad} \hyphenation{ob-te-ner}
\hyphenation{des-ple-gar}
\hyphenation{original}
\hyphenation{geo-gra-fi-ca} \hyphenation{do-cu-men-to}
\hyphenation{do-cu-men-tos}
\hyphenation{e-va-lua-da}\hyphenation{pro-pues-ta}
\hyphenation{tu-ris-mo}\hyphenation{pro-pues-to}
\hyphenation{ma-yo-res}\hyphenation{des-cri-be}
\hyphenation{re-le-van-cia}\hyphenation{pro-ble-ma}
\hyphenation{ma-ne-ra}\hyphenation{co-rres-pon-de}
\hyphenation{pro-pues-to}\hyphenation{re-tro-a-li-men-ta-cion}
\hyphenation{si-mi-li-tud}\hyphenation{tradicional}\hyphenation{ordenamiento}
\hyphenation{pu-bli-ca-cio-nes}\hyphenation{a-tri-bu-tos}
\hyphenation{re-cu-pe-rar}\hyphenation{re-cu-pe-ra-das}

% espaciado : 1 y 1/2
\onehalfspacing

% establecemos la enumeración para las equaciones y figuras
\numberwithin{equation}{section}
\renewcommand{\thefigure}{\thechapter.\arabic{figure}}

\setlength{\arrayrulewidth}{1pt}
\setlength{\doublerulesep}{0mm}

\sloppy
\frenchspacing

% redefinimos el comando para incrustar dos páginas vacías
\newcommand{\clearemptydoublepage}{\newpage{\pagestyle{empty}\cleardoublepage}}


%#################################################################################
% inicio del documento
\begin{document}

% incluimos el pdf con la portada (sólo la página 1) (la azul fea) como primera página de este documento
% nota: debes modificar y compilar de manera separada el archivo portada.tex, luego compilar este documento para que incluya la versión modificada del PDF (portada.pdf)
%\includepdf[pages={1}]{portada.pdf}
\includepdf[pages=1]{PORTADA.pdf}
\clearemptydoublepage

% incluimos el .tex que tiene la información para la contraportada (puede llamarse como desees, pero recuerda poner el nombre sin extensión)
\include{contraportada}                  %%%%%%%%
\clearemptydoublepage


\frontmatter
\pagestyle{fancyplain}



%###############################################################################33
% redefinimos los márgenes para el documento
\setlength{\topmargin}{0.0cm}
\setlength{\headsep}{0.5cm}
\setlength{\headheight}{0.5cm}

\setlength{\marginparsep}{0mm}
\setlength{\marginparwidth}{0cm}

\setlength{\footskip}{1.5cm}

  \setlength\paperheight {11in}
  \setlength\paperwidth  {8.5in}

  \setlength{\textwidth}{6in}
  \setlength{\textheight}{8.5in}

  \setlength{\marginparwidth}{0cm}
  \setlength{\voffset}{0.0in}
  \setlength{\hoffset}{0.0in}

 %\setlength{\evensidemargin}{0.32} % margen paro lvera
  \setlength{\evensidemargin}{0in} % margen par
  \setlength{\oddsidemargin}{0.32in}  % margen impar
 % \setlength{\oddsidemargin}{0in}  % margen impar



%###################################################################################
% especificamos el nombre en español para cada sección estándar de la tesis
\def\contentsname{\'{I}ndice General}
\def\contentsname{Tabla de Contenido}
\def\listfigurename{Lista de Figuras}
\def\listtablename{Lista de Tablas}
\def\bibname{Bibliograf�a}
\def\indexname{\'{I}ndice Alfabetico}
\def\figurename{\bf \scriptsize Figura}
\def\tablename{\bf \scriptsize Tabla}
\def\partname{Parte}
\def\chaptername{Cap\'{i}tulo~\thechapter}
\def\appendixname{Ap\'{e}ndice}
\def\abstractname{Resumen}


% presentación de título y capítulo
\renewcommand{\chaptermark}[1]{\markboth{#1}{}}
% número y título de sección
\renewcommand{\sectionmark}[1]{\markright{\thesection~~ #1}}




\lhead[\fancyplain{}{\small \thepage}]{\fancyplain{}{\scshape
\scriptsize \rightmark}} %\rightmark = información sobre el capítulo

\rhead[\fancyplain{}{\scshape \scriptsize
\leftmark}]{\fancyplain{}{\small \thepage}} %\leftmark = información sobre la sección

\lfoot[\fancyplain{}{\Escuela}]{\fancyplain{}{}}
\cfoot[\fancyplain{\tstamp}{}]{\fancyplain{\tstamp}{\TEMA}}
\rfoot[\fancyplain{}{\Uni}]{\fancyplain{}{}}

\renewcommand{\headrulewidth}{0.4pt}
\renewcommand{\footrulewidth}{\headrulewidth}

% cambiar el formato para los títulos
\newcommand{\bigrule}{\titlerule[0.5mm]}

\titleformat{\chapter}[display]
{\bfseries \Huge } {
 \titlerule
 \filleft\Large\chaptertitlename
 }
{0mm} {\filleft}
 [\vspace{0.5mm}\bigrule]


% incluimos los .tex de cada uno de las secciones previas al verdadero documento (recuerda: sólo debe ir el nombre del .tex)
% los puntos .tex van sin preámbulo, es decir que sólo va el texto correspondiente y las instrucciones para definir título, secciones, subsecciones, tablas, etc.
%IMD PNA http://na.support.keysight.com/pna/help/latest/Applications/Swept_IMD_Configure_External_Source_and_Combiner.htm

\vspace*{\fill}
\begin{center}
	\textbf{\huge Dedicatoria}
\end{center}

\begin{center}
Este trabajo se lo dedico a todos los m�sicos que sean apasionados por la tecnolog�a, y ven en ella una forma para seguir evolucionando musicalmente. 
\end{center}
\vspace*{\fill}

%%-----------------------------------------------------------------------


%IMD PNA http://na.support.keysight.com/pna/help/latest/Applications/Swept_IMD_Configure_External_Source_and_Combiner.htm
\begin{center}
\textbf{\huge Agradecimientos}
\end{center}

Primeramente agradezco a Dios por permitirme haber concluido un ciclo mas en mi formaci�n profesional, cada paso que doy siempre es gracias a �l.

Agradezco a mi madre cuya fuerza de voluntad fue capaz de sacar adelante a una familia completa, ella me ense�o el valor del trabajo y del estudio. Gracias a ella que me brindo un consejo cuando lo necesite, un hombro de apoyo cuando flaqueaba y una reprimenda cuando no estaba asciendo las cosas correctamente. Gracias a su amor incondicional soy quien soy ahora.

A mi novia que estuvo conmigo en �pocas dif�ciles en mi vida, cuando pens� que no podr�a terminar con mis estudios por todas las responsabilidades que hab�a adquirido, ella siempre estuvo ah� para animarme a seguir adelante.

Mis hermanas las cuales siempre me dieron palabras de aliento y siempre estuvieron preocupadas por mi, en todo momento, agradezco su apoyo incondicional.

Al CONACYT porque me brindo el apoyo econ�mico para poder terminar este grado de estudio.

A mis maestros que tuvieron la suficiente paciencia y dedicaci�n para transmitirnos parte de su conocimiento, gracias a ellos pude concluir con este grado, ganando el suficiente conocimiento para poder aplicarlo en el campo laboral.

A mi Asesor el Maestro Juan Antonio Vega Fern�ndez, el cual me gu�o a lo largo de esta tesis para realizar un trabajo que pueda dejar algo de conocimiento a la comunidad, agradezco la paciencia que me tuvo y el tiempo que me dedico para la revisi�n de este proyecto.

A mis compa�eros de generaci�n los cuales me ense�aron much�simas cosas, fue muy enriquecedor contar con personas que a pesar de que hab�an vivido cosas muy diferentes, ten�an en la mente seguir super�ndose escolarmente y profesionalmente hablando.

Finalmente agradezco a la UAG, escuela que me dejo tanto aprendizaje y vivencias. A todo el personal administrativo de esta instituci�n que me guiaron a travez de todos los tramites necesarios para concluir esta etapa.

%%-----------------------------------------------------------------------


\include{resumen_1}
\include{abstract_2}


% por si quieres incuir hasta un tercer nivel en la tabal de contenidos
%\setcounter{tocdepth}{3}

% creamos la tabla de contenidos y las listas de figuras y tablas
\tableofcontents         %%
\listoffigures           %%
\listoftables            %%

\mainmatter
\renewcommand{\chaptermark}[1]{\markboth{\thechapter.~~#1}{}}

%########################################################################################
% ahora si incluimos cada uno  de los .tex correspondientes a los capítulos de la tesis. Es más claro dividir los capítulos en .tex aunque puedes poner tooodo el texto aquí.
% los puntos .tex van sin preámbulo, es decir que sólo va el texto correspondiente y las instrucciones para definir título, secciones, subsecciones, tablas, etc.

\chapter{Introducci�n}


\section{Descripci�n del Problema}
���

\section{Objetivos}
\subsection{Objetivo General}Objetivo General.


\subsection{Objetivos Espec�ficos}
\begin{enumerate}
  \item Objetivo espec�fico 1
  \item Objetivo espec�fico 2
\end{enumerate}

\section{Justificaci�n}

\section{Delimitaci�n}

\section{Organizaci�n de la Tesis}










%IMD PNA http://na.support.keysight.com/pna/help/latest/Applications/Swept_IMD_Configure_External_Source_and_Combiner.htm
\chapter{Redes neuronales}



%%-----------------------------------------------------------------------

Las redes neuronales artificiales son un conjunto de neuronas creadas artificialmente para el desarrollo de inteligencia artificial en una computadora.
 
Estas redes est�n basadas en las redes biol�gicas del cerebro humano, modelando todos los factores biol�gicos de las neuronas.

Debido a su dise�o las redes son capaces de aprender de la experiencia, de generalizar de casos anteriores a nuevos casos, de abstraer caracter�sticas esenciales a partir de entradas que representan en ocasiones informaci�n irreverente.

La capacidad de aprendizaje adaptativo es una caracter�stica fundamental de las redes neuronales y les permiten llevar a cabo ciertas tareas mediante un entrenamiento previo, pueden aprender a diferenciar patrones y generalizar a partir de estos. Son considerados sistemas din�micos ya que son capaces de adaptarse a nuevas condiciones de entrada.

Tienen una alta tolerancia a fallos ya que son capaces de detectar patrones aun cuando estos patrones posean ruido, distorsi�n o simplemente est�n incompletos. Estos programas son capaces de seguir funcionando incluso si parte de la red presente fallas.

La informaci�n se almacena de forma distribuida en las conexiones de las neuronas, provocando redundancia de informaci�n, es decir se guardara sus valores en base a la funci�n de activaci�n que posee cada neurona, de esta manera si una neurona es destruida o presenta fallas, las otras neuronas podr�n aprender la informaci�n de la neurona que fallo.

Las neuronas humanas poseen diferentes secciones:

\begin{figure}[H]
	\centerline{\includegraphics[width=8cm]{estruc_neu.png}}
	\caption{Estructura general de una neurona}
	\label{fig:estruc_neu}
\end{figure}

En una neurona artificial se busca la emulaci�n de las principales secciones de una neurona las cuales son:

\begin{itemize}
	\item \textbf{Cuerpo.-} Se encarga de producir un impulso el�ctrico en base a las entradas de la neurona.
	\item \textbf{Dendritas.-} son filamentos capaces de crear conexiones con otras neuronas.
	\item \textbf{Ax�n.-} Es el encargado de transmitir el impulso el�ctrico generado por el cuerpo.
\end{itemize}

A continuaci�n se muestra una imagen de como lucir�a una neurona artificial:

\begin{figure}[H]
	\centerline{\includegraphics[width=13cm]{neu_art.png}}
	\caption{Neurona Artificial}
	\label{fig:neu_art}
\end{figure}

Esta neurona posee las siguientes secciones:

\begin{itemize}
	\item \textbf{$X_1$, $X_2$, \ldots, $X_n$ .-} Son las entradas de la neurona.
	\item \textbf{$W_1$, $W_2$, \ldots, $W_n$.-} Pesos espec�ficos que tendr� cada entrada, esto hace que las entradas no valgan lo mismo ponderadamente.
	\item \textbf{Funci�n de red.-} Esta es una funci�n de sumatoria de las entradas.
	\item \textbf{Funci�n de activaci�n.-} Si la suma de las entradas es mayor o igual que el umbral definido por esta funci�n se tendr� una se�al a la salida.
\end{itemize}

La salida de la neurona viene dada por esta ecuaci�n:

\begin{equation}
y_j = f(\sum_{i=1}^{n}(W_{ij}x_i + \theta_j))
\label{ec:neu_out}
\end{equation}

Una red neuronal no es m�s que la interconexi�n de varias neuronas artificiales, en la cual podemos identificar al menos tres secciones:

\begin{itemize}
	\item \textbf{Capa de entrada.-} En esta capa se procesan todas las entradas, y si estas entradas son capaces de excitar las neuronas de esta capa se producir� una se�al de salida.
	\item \textbf{Capa oculta.-} En esta capa se encuentran las neuronas encargadas del aprendizaje de la red.
	\item \textbf{Capa de salida.-} Esta neurona o neuronas de salida tendr�n la salida del sistema.
\end{itemize}

La forma en que las redes aprenden es mediante la modificaci�n de los pesos de las entradas.

Las redes neuronales artificiales han ido evolucionando con el paso del tiempo, hoy en d�a existen muchos modelos de redes neuronales, todas ellas con ventajas y desventajas si son comparadas entre ellas.

\section{Aprendizaje de las redes neuronales}

%TODO: Rf
El procedimiento utilizado para llevar a cabo el proceso de aprendizaje en una red neuronal se denomina entrenamiento.

El problema de aprendizaje en las redes neuronales se formula en t�rminos de la minimizaci�n de la funci�n de error (o p�rdida) asociada.

Normalmente, esta funci�n est� compuesta por dos t�rminos, uno que eval�a c�mo se ajusta la salida de la red neuronal al conjunto de datos de que disponemos, y que se denomina t�rmino de error, y otro que se denomina t�rmino de regularizaci�n, y que se utiliza para evitar el sobreaprendizaje por medio del control de la complejidad efectiva de la red neuronal.

Por supuesto, el valor de la funci�n de error depende por completo de los par�metros de la red neuronal: los pesos sin�pticos entre neuronas, y los bias asociados a ellas, que, como suele ser ya habitual, se pueden agrupar adecuadamente en un �nico vector de peso de la dimensi�n adecuada, que denotaremos por $w$. En este sentido, podemos escribir $f(w)$ para indicar que el valor del error que comete la red neuronal depende de los pesos asociados a la misma. Con esta formalizaci�n, nuestro objetivo es encontrar el valor $w^{*}$ para el que se obtiene un m�nimo global de la funci�n $f$, convirtiendo el problema de aprendizaje en un problema de optimizaci�n.

En general, la funci�n de error es una funci�n no lineal, por lo que no disponemos de algoritmos sencillos y exactos para encontrar sus m�nimos. En consecuencia, tendremos que hacer uso de una b�squeda a trav�s del espacio de par�metros que, idealmente, se aproxime de forma iterada a a un (error) m�nimo de la red para los par�metros adecuados.

De esta forma, se comienza con una red neuronal con alg�n vector inicial de par�metros (a menudo elegido al azar), a continuaci�n se genera un nuevo vector de par�metros, esperando que con ellos la funci�n de error se reduzca (aunque dependiendo del m�todo elegido, no es obligatorio, y temporalmente se puede admitir un empeoramiento del error siempre y cuando conduzca a una disminuci�n posterior m�s acusada). Este proceso se repite, normalmente, hasta haber reducido el error bajo un umbral tolerable, o cuando se satisfaga una condici�n espec�fica de parada.

El Descenso del Gradiente es el algoritmo de entrenamiento m�s simple y tambi�n el m�s extendido y conocido. Solo hace uso del vector gradiente, y por ello se dice que es un m�todo de primer orden.

Este m�todo para construir el punto $w_{i+1}$ a partir de $w_{i}$ se traslada este punto en la direcci�n de entrenamiento $d_i=-g_i$. Es decir:

\begin{equation}
w_{i+1} = w_{i} - g_{i}v_{i}
\label{ec:grad}
\end{equation}

Donde el par�metro $v$ se denomina tasa de entrenamiento, que puede fijarse a priori o calcularse mediante un proceso de optimizaci�n unidimensional a lo largo de la direcci�n de entrenamiento para cada uno de los pasos (aunque esta �ltima opci�n es preferible, a menudo se usa un valor fijo, $v_{i}=v$ con el fin de simplificar el proceso).

Aunque es muy sencillo, este algoritmo tiene el gran inconveniente de que, para funciones de error con estructuras con valles largos y estrechos, requiere muchas iteraciones. Se debe a que, aunque la direcci�n elegida es en la que la funci�n de error disminuye m�s r�pidamente, esto no significa que necesariamente produzca la convergencia m�s r�pida.

Por ello, es el algoritmo recomendado cuando tenemos redes neuronales muy grandes, con muchos miles de par�metros, ya que s�lo almacena el vector gradiente (de tama�o n).
%http://www.cs.us.es/~fsancho/?e=165
%Rf

\section{Deep Learning}

Deep Learning usa redes neuronales con muchas capas para lograr aprendizajes m�s complejos. Este comportamiento asemeja la forma en que el cerebro humano toma decisiones, el cual usa la interconexi�n de varias capas de neuronas para realizar actividades complejas.
Dentro de las redes de Deep Learning se tienen 2 tipos muy usados en la actualidad:

\begin{itemize}
	\item \textbf{Redes convolucionales (CNN).-} Este tipo de redes usa la convoluci�n en varias de sus capaz para lograr el procesamiento de par�metros que se pueden representar en un espacio $R^2$, un ejemplo claro de esto son las im�genes y v�deos, por lo tanto si se quiere hacer una clasificaci�n o reconocimiento de im�genes, este tipo de redes nos proporcionan una buena herramienta de procesamiento.
	\item \textbf{Redes recurrentes (RNN).-} Este tipo de redes son muy usadas cuando se busca analizar una secuencia de datos, estas redes poseen memoria y una retroalimentaci�n de la salida a la entrada.  
\end{itemize}

\section{Redes neuronales recurrentes}

%TODO: Rf
La idea detr�s de las RNN es hacer uso de la informaci�n secuencial. En una red neuronal tradicional suponemos que todas las entradas (y salidas) son independientes entre s�. Pero para muchas tareas eso es una muy mala idea. Si quieres predecir la siguiente palabra en una oraci�n, es mejor que conozcas qu� palabras vienen antes. Las RNN se llaman recurrentes porque realizan la misma tarea para cada elemento de una secuencia, y la salida depende de los c�lculos previos. Otra forma de pensar acerca de las RNN es que tienen una "memoria" que captura informaci�n sobre lo que se ha calculado hasta ahora. En teor�a, los RNN pueden hacer uso de la informaci�n en secuencias arbitrariamente largas, pero en la pr�ctica se limitan a mirar hacia atr�s solo unos pocos pasos.

La decisi�n de una red recurrente alcanzada en el paso de tiempo t-1 afecta la decisi�n que alcanzar� un momento m�s tarde en el paso de tiempo t (Figura \ref{fig:red_rec}). Entonces, las redes recurrentes tienen dos fuentes de entrada, el presente y el pasado reciente, que se combinan para determinar c�mo responden a los datos nuevos, de forma similar a como lo hacemos en la vida.

\begin{figure}[H]
	\centerline{\includegraphics[width=3cm]{red_rec.png}}
	\caption{Redes recurrentes}
	\label{fig:red_rec}
\end{figure}

Esa informaci�n secuencial se conserva en el estado oculto de la red recurrente, que logra abarcar muchos pasos de tiempo a medida que avanza para afectar el procesamiento de cada nuevo ejemplo. Est� encontrando correlaciones entre eventos separados por muchos momentos, y estas correlaciones se llaman "dependencias a largo plazo", porque un evento en el tiempo depende de, y es una funci�n de, uno o m�s eventos que vinieron antes. Una forma de pensar acerca de las RNN es esta: son una forma de compartir pesos a lo largo del tiempo.

As� como la memoria humana circula invisiblemente dentro de un cuerpo, afectando nuestro comportamiento sin revelar su forma completa, la informaci�n circula en los estados ocultos de las redes recurrentes.

Describiremos el proceso de llevar la memoria hacia adelante matem�ticamente:

\begin{equation}
h_t = \phi(Wx_t + Uh_{t-1})
\label{ec:mem_neu}
\end{equation}

El estado oculto en el paso de tiempo $t$ es $h_t$. Es una funci�n de la entrada al mismo tiempo paso $x_t$, modificada por una matriz de ponderaci�n W (como la que usamos para las redes feedforward) agregada al estado oculto del paso de tiempo anterior $h_{t-1}$ multiplicado por su propio estado oculto matriz U de estado oculto, tambi�n conocida como matriz de transici�n y similar a una cadena de Markov. Las matrices de peso son filtros que determinan la importancia de acuerdo tanto con la entrada actual como con el estado oculto pasado. El error que generan volver� a trav�s de la propagaci�n inversa y se usar� para ajustar sus ponderaciones hasta que el error no pueda bajar m�s.

Debido a que este ciclo de retroalimentaci�n ocurre en cada paso de la serie, cada estado oculto contiene rastros no solo del estado oculto anterior, sino tambi�n de todos los que precedieron a $h_{t-1}$ mientras la memoria pueda persistir.

\subsection{LSTM}

A mediados de los a�os 90, los investigadores alemanes Sepp Hochreiter y Juergen Schmidhuber propusieron una variaci�n de la red recurrente con las denominadas unidades de memoria a largo plazo, o LSTM, como una soluci�n al problema del gradiente de fuga.

Los LSTM ayudan a preservar el error que se puede volver a propagar a trav�s del tiempo y las capas. Al mantener un error m�s constante, permiten que las redes recurrentes contin�en aprendiendo durante muchos pasos de tiempo (m�s de 1000), abriendo as� un canal para vincular causas y efectos de forma remota. Este es uno de los desaf�os centrales para el aprendizaje autom�tico y la IA, ya que los algoritmos se enfrentan con frecuencia a entornos en los que las se�ales de recompensa son dispersas y diferidas, como la vida misma.

Los LSTM contienen informaci�n fuera del flujo normal de la red recurrente en una celda cerrada. La informaci�n puede almacenarse, escribirse o leerse desde una celda, al igual que los datos en la memoria de una computadora. La c�lula toma decisiones sobre qu� almacenar y cu�ndo permitir las lecturas, escrituras y borraduras, a trav�s de puertas que se abren y cierran. Sin embargo, a diferencia del almacenamiento digital en computadoras, estas puertas son an�logas, implementadas con la multiplicaci�n de elementos por sigmoides, que est�n todas en el rango de 0-1. Siendo Analogica tiene la ventaja sobre digital de ser diferenciable y, por lo tanto, adecuado para la propagaci�n inversa.

Esas puertas act�an sobre las se�ales que reciben, y de forma similar a los nodos de la red neuronal, bloquean o transmiten informaci�n en funci�n de su fuerza e importaci�n, que filtran con sus propios conjuntos de ponderaciones. Esos pesos, como los pesos que modulan los estados de entrada y ocultos, se ajustan a trav�s del proceso de aprendizaje de redes recurrentes. Es decir, las c�lulas aprenden cu�ndo permiten que los datos entren, salgan o se eliminen a trav�s del proceso iterativo de hacer conjeturas, volver a propagar el error y ajustar los pesos mediante el descenso del gradiente.
% https://skymind.ai/wiki/lstm
%Rf

\begin{figure}[H]
	\centerline{\includegraphics[width=10cm]{lstm.png}}
	\caption{Redes LSTM}
	\label{fig:lstm}
\end{figure}

\section{Optimizadores}

% https://towardsdatascience.com/types-of-optimization-algorithms-used-in-neural-networks-and-ways-to-optimize-gradient-95ae5d39529f

Los algoritmos de optimizaci�n nos ayudan a minimizar (o maximizar) una funci�n de objetivo $E(x)$, que es simplemente una funci�n matem�tica que depende de los par�metros internos de aprendizaje del modelo que se utilizan para calcular los valores objetivo $(Y)$ a partir de Conjunto de predictores $(X)$ utilizados en el modelo. 

Los algoritmos de optimizaci�n estan en 2 categor�as principales:

\begin{itemize}
	\item  \textbf{Algoritmos de optimizaci�n de primer orden: } estos algoritmos minimizan o maximizan una funci�n de p�rdida $E(x)$ utilizando sus valores de gradiente con respecto a los par�metros. El algoritmo de optimizaci�n de primer orden m�s utilizado es Descenso por Gradiente. La derivada de primer orden nos dice si la funci�n est� disminuyendo o aumentando en un punto en particular.
	\item  \textbf{Algoritmos de optimizaci�n de segundo orden: } estos algoritmos utilizan la derivada de segundo orden, que tambi�n se llama Hessiana para minimizar o maximizar la funci�n de error $E(x)$. La Hessiana es una matriz de derivados parciales de segundo orden. Dado que la segunda derivada es costosa de calcular, el segundo orden no se usa mucho. La derivada de segundo orden nos dice si la primera derivada est� aumentando o disminuyendo, lo que sugiere la curvatura de la funci�n.
\end{itemize}

Un gradiente es simplemente un vector que es una generalizaci�n multivariable de una derivada $(\frac{dy}{dx})$ en funciones multivariables.

\subsection{Descenso por gradiente}

El descenso por gradiente es la t�cnica mas usada para entrenar y optimizar sistemas inteligentes. Es utilizada para realizar la actualizaci�n de los pesos en modelos de redes neuronales y minimizar la funci�n de error $E(x)$, la forma en que actualiza los par�metros es utilizando la siguiente ecuaci�n:

\begin{equation}
\theta = \theta - \eta\nabla J(\theta)
\label{ec:des_grad}
\end{equation}

Donde $\eta$ es la tasa de aprendizaje, y $\nabla J(\theta)$ es el gradiente de la funci�n de error $J(\theta)$.
El descenso por gradiente es mayormente usado para hacer la actualizaci�n de los pesos en una red neuronal.
%IMD PNA http://na.support.keysight.com/pna/help/latest/Applications/Swept_IMD_Configure_External_Source_and_Combiner.htm
\chapter{Teor�a musical}



%%-----------------------------------------------------------------------


%\section{}

La teor�a musical es el estudio de los elementos que conforman la m�sica. En esta teor�a se analizan todos los sonidos involucrados para la creaci�n, an�lisis y composici�n musical.

La m�sica es un arte que se basa en 2 elementos: 

\begin{itemize} [noitemsep]
	\item Los sonidos
	\item Los silencios
\end{itemize}

Los sonidos tienen diferentes propiedades las cuales se describen a continuaci�n:

\begin{itemize}
	\item \textbf{Altura.-} un sonido puede ser agudo, medio o grave, dependiendo de la altura de su nota.
	\item \textbf{Duraci�n.-} un sonido debe de tener una duraci�n la cual se expresa en unidades de tiempo.
	\item \textbf{Intensidad.-} esto se refiere al volumen del sonido, puede ser d�bil o fuerte.
	\item \textbf{Timbre.-} se le conoce como timbre o color del sonido a como un sonido con la misma nota suena diferente dependiendo del instrumento usado para su interpretaci�n.  
\end{itemize}

Los silencios a su vez su �nica propiedad intr�nseca es la duraci�n, es decir en un silencio lo �nico que se mide es la duraci�n del mismo.

\section{Composici�n musical}

La composici�n musical esta catalogado como un arte que tiene como objeto crear nuevas piezas musicales.

Un m�sico puede optar por varios caminos para la creaci�n de su obra, existen m�sicos que se basan en su simple sentido com�n y crean canciones liricamente, sin embargo el proceso formal de composici�n involucra todos los conceptos musicales b�sicos de la teor�a musical, los cuales se describir�n a continuaci�n.

\subsection{Notas musicales}

Las notas es un sistema que se usa para la representaci�n de los diferentes sonidos en la m�sica. En el mundo occidental se usa un sistema de 12 notas, los cuales pueden ser repetidos con diferentes alturas para generar una gama muy amplia de sonidos.
Dentro de este sistema de 12 notas tenemos las notas naturales y las notas con alteraciones.
Las notas naturales son:

Do, Re, Mi, Fa, Sol, La, Si   (7 notas)

Las alteraciones no son m�s que agregar o quitar medios tonos a una nota, para eso se usan los siguientes s�mbolos:

\begin{itemize}
	\item \textbf{\# (sostenido).-} se agrega $1/2$ tono a una nota.
	\item \textbf{x (doble sostenido).-} Se agrega 1 tono a una nota.
	\item \textbf{$\flat$ (bemol).-} se disminuye $1/2$ tono a una nota.
	\item \textbf{$\flat\flat$ (doble bemol).-} Se disminuye 1 tono a una nota.  
\end{itemize}

Usando los s�mbolos anteriores podemos definir las siguientes notas con alteraciones:

Do\#, Re\#, Fa\#, Sol\#, La\#  (5 notas)

Como se puede observar tanto Mi y Si no tienen sonidos con alteraciones ascendentes, ya que los sonidos producidos por estas alteraciones son igual al de las notas consecutivas, a este tipo de sonidos iguales se les conoce como notas enarm�nicas.

Por ejemplo:

\begin{itemize}
	\item Mi\# sonar�a exactamente igual que un Fa.
	\item Si\# sonar�a exactamente igual que un Do.  
\end{itemize}

Tambi�n las notas con alteraciones pueden ser representadas usando el s�mbolo de bemol (b), es decir restando medio tono a una nota. En este caso se tendr�an las notas con alteraciones de la siguiente manera:

Re$_\flat$, Mi$_\flat$, Sol$_\flat$, La$_\flat$, Si$_\flat$      (5 notas)

Como se mencion� anteriormente las notas que en sonido son exactamente iguales pero en nomenclatura son diferentes se conocen como notas enarm�nicas, de tal manera que las 5 notas con alteraciones que se describieron se pueden hacer una comparaci�n del sonido de las mismas, siendo as� se tiene:

\begin{itemize}
	\item Do\#  suena exactamente igual que Re$_\flat$.
	\item Re\# suena exactamente igual que Mi$_\flat$. 
	\item Fa\# suena exactamente igual que Sol$_\flat$. 
	\item Sol\# suena exactamente igual que La$_\flat$. 
	\item La\# suena exactamente igual que Si$_\flat$.   
\end{itemize}

Podemos ver la relaci�n de tonos y semitonos (1/2 tonos) en la siguiente figura:

\begin{figure}[H]
	\centerline{\includegraphics[width=8cm]{tonos.png}}
	\caption{Relacion de tonos y semitonos}
	\label{fig:tonos}
\end{figure}

Como se puede observar entre Mi y Fa existe un semitono, al igual que entre Si y Do.

El pentagrama es un sistema de cinco l�neas y cuatro espacios para escribir m�sica:

\begin{figure}[H]
	\centerline{\includegraphics[width=8cm]{pentagrama.png}}
	\caption{Pentagrama}
	\label{fig:pentagrama}
\end{figure}

Este sistema puede tener diferentes elementos dentro de los principales tenemos:

\begin{itemize} [noitemsep]
	\item Clave.
	\item Comp�s.
	\item Armadura. 
	\item Notas.
\end{itemize}

\subsection{Claves musicales}

Las claves musicales se usan para darle nombre y altura a las notas musicales dentro de un pentagrama.

Las claves m�s usadas en la m�sica son: la clave de Sol, la clave de Fa y la clave de Do.

La clave de Sol se usa en instrumentos agudos como la guitarra, viol�n, entre otros. Esta clave normalmente se escribe empezando en la segunda l�nea del pentagrama para formar una especie de G. El hecho de que esta clave se escriba en la segunda l�nea establece que esa l�nea ser� llamada como el nombre de la clave, en este caso Sol:

\begin{figure}[H]
	\centerline{\includegraphics[width=4cm]{clave_sol.png}}
	\caption{Clave de Sol en 2da. Linea}
	\label{fig:clave_sol}
\end{figure}

La clave de Fa es usada en instrumentos m�s graves, tales como el contrabajo, el bajo, etc. Normalmente se escribe empezando en la cuarta l�nea, de tal manera que esta l�nea tomara el nombre de Fa:

\begin{figure}[H]
	\centerline{\includegraphics[width=4cm]{clave_fa.png}}
	\caption{Clave de Fa en 4ta. Linea}
	\label{fig:clave_fa}
\end{figure}

La clave de Do es usada com�nmente para las voces, y esta clave se escribe en diferentes l�neas dependiendo del timbre de la voz, para un tenor se usa la clave de Do en 4ta l�nea, mientras que para un soprano normalmente la clave se usa en 3ra o 2da l�nea:

\begin{figure}[H]
	\centerline{\includegraphics[width=8cm]{clave_do.png}}
	\caption{Clave de Do en 4ta. y 3ra. linea}
	\label{fig:clave_do}
\end{figure}

A partir de estas claves se le puede poner nombre a las diferentes l�neas y espacios del pentagrama, por ejemplo en la clave de sol en segunda l�nea, el pentagrama quedar�a de la siguiente forma:

\begin{figure}[H]
	\centerline{\includegraphics[width=8cm]{notas_clave_sol.png}}
	\caption{Nombre de las notas en un pentagrama con clave de Sol}
	\label{fig:notas_clave_sol}
\end{figure}

\subsection{Compases y tiempo}

El comp�s se puede definir como la unidad m�trica de la m�sica, es la que nos dice que tiempo llevara la canci�n.

Existen una infinidad de compases los cuales podemos clasificar en dos grandes grupos:

\begin{itemize} [noitemsep,nolistsep]
	\item Compases regulares.
	\item Compases irregulares.
\end{itemize}

Los compases regulares est�n regidos por formas regulares las cuales dictan el tiempo, mientras que en los compases irregulares hay que determinar la base de tiempo de forma indirecta.

Cada nota puede tener un valor de tiempo y este valor estar� determinado por el comp�s que se est� utilizando, por ejemplo en un comp�s de 4/4 las figuras musicales tendr�n los siguientes valores:

\begin{figure}[H]
	\centerline{\includegraphics[width=8cm]{dura_notas.png}}
	\caption{Duraci�n de las notas musicales}
	\label{fig:dura_notas}
\end{figure}

Como se puede observar para cada figura musical existe su silencio correspondiente, el cual tendr� el mismo valor solo que este caso no se producir� sonido alguno.

Podemos tener tambi�n los valores relativos de estas notas respecto a otras notas:

\begin{figure}[H]
	\centerline{\includegraphics[width=8cm]{valor_rel.png}}
	\caption{Valor relativo de las notas}
	\label{fig:valor_rel}
\end{figure}

De tal manera que en un comp�s tendremos dos datos los cuales se describen a continuaci�n:

\begin{figure}[H]
	\centerline{\includegraphics[width=8cm]{compas.png}}
	\caption{Compas}
	\label{fig:compas}
\end{figure}

Se puede ver que este compas tendr� 2 pulsos y el valor de cada pulso ser� de un tiempo de negra, ya que el valor relativo de la negra respecto a la redonda es de $1/4$.

Otro punto a considerar en los compases es que estos tienen tiempos fuertes, semifuertes y d�biles. La cuadratura de una armon�a usa estos tiempos para indicar los cambios que se deben de hacer.

Este es un ejemplo de los tiempos en compases de $4/4$, $3/4$ y $2/4$: 

\begin{figure}[H]
	\centerline{\includegraphics[width=8cm]{tiempos_fd.png}}
	\caption{Tiempos fuertes y d�biles}
	\label{fig:tiempos_fd}
\end{figure}

\subsection{Escalas}

Las escalas son una serie de notas musicales que siguen un orden establecido por intervalos desde una nota base. En el mundo de la m�sica hay una infinidad de escalas pero todas parten de la escala mayor de Do:

\begin{figure}[H]
	\centerline{\includegraphics[width=8cm]{escala.png}}
	\caption{Escala de Do Mayor}
	\label{fig:escala}
\end{figure}

En la escala mayor se tiene los siguientes intervalos: tono, tono, semitono, tono, tono, tono, semitono, si esta f�rmula la aplicamos con las otras notas musicales podremos construir todas las escalas mayores.

Por ejemplo la escala de sol mayor quedar�a de la siguiente manera:

\begin{figure}[H]
	\centerline{\includegraphics[width=8cm]{escala_sol.png}}
	\caption{Escala de Sol Mayor}
	\label{fig:escala_sol}
\end{figure}

Para poder completar el tono completo de la f�rmula de la escala mayor entre Mi y Fa se tuvo que poner una alteraci�n en la nota de Fa.

La �nica escala mayor que no posee alteraciones es la escala mayor de Do, de ah� en m�s todas las dem�s escalas mayores tendr�n al menos una alteraci�n.

Las escalas menores parten de la escala mayor obteniendo su sexta nota y siguiendo las mismas notas de la escala mayor. Estas escalas menores se les conocen como menores relativas, ya que se basan en una escala mayor para su formaci�n.

Por ejemplo la escala relativa de Do mayor seria La menor y esta escala posee exactamente las mismas notas que la escala mayor solamente que empieza desde la nota de La:

En este caso las notas son: La, Si, Do, Re, Mi, Fa, Sol, La.

\begin{figure}[H]
	\centerline{\includegraphics[width=8cm]{escala_la_men.png}}
	\caption{Escala de La menor}
	\label{fig:escala_la_men}
\end{figure}

Esta relatividad puede ser usada con todas las escalas mayores para obtener sus relativas menores.

A pesar que las escalas mayores y menores poseen las mismas notas no deben ser nunca confundidas ya que el sonido final producido es muy diferente, mientras las escalas mayores se usan para canciones se puede decir hasta cierto punto alegres, las escalas menores normalmente acompa�an melod�as melanc�licas, esto no es en todos los casos.

\subsection{Armaduras}

Todas las alteraciones de una escala se pueden juntar al inicio del pentagrama para dar lugar a lo que se conoce como armaduras.

Estas armaduras indicaran que todas las notas que poseen alteraciones las mantendr�n a lo largo de toda la canci�n.

Debido a que las escalas mayores y sus relativas menores poseen las mismas alteraciones por lo tanto comparten tambi�n la misma armadura:

\begin{figure}[H]
	\centerline{\includegraphics[width=6cm]{armadura_sost.png}}
	\caption{Armaduras relativas con sostenidos}
	\label{fig:armadura_sost}
\end{figure}

Tambi�n podemos tener armaduras con bemoles:

\begin{figure}[H]
	\centerline{\includegraphics[width=6cm]{armadura_bem.png}}
	\caption{Armaduras relativas con bemoles}
	\label{fig:armadura_bem}
\end{figure}

\subsection{Tonalidades}

La tonalidad de una canci�n se basa en la escala base de la canci�n y esta la podemos averiguar viendo la armadura que posee la canci�n, aunque como se vio anteriormente las escalas mayores y sus relativas poseen la misma armadura, as� que antes de definir la tonalidad en base a la armadura tambi�n se debe de hacer un an�lisis de la interacci�n de las notas en la canci�n.

Dentro de las tonalidades tenemos tonalidades mayores y menores, por ejemplo si vemos un pentagrama el cual no posee alteraciones podr�amos asumir que la canci�n esta en Do mayor o en La menor, el siguiente paso ser�a ver la interacci�n de las notas en la canci�n para determinar correctamente la tonalidad.

\subsection{Melod�as y Armon�as}

Una melod�a es una sucesi�n de notas de forma ascendente o descendente que llevan cierta cordura. 

Las melod�as suelen estar formadas por frases y generalmente se repiten a lo largo de una canci�n variando algunas notas intermedias. En este caso se trata de una sucesi�n de notas que no son tocadas al mismo tiempo sino que una nota es tocada despu�s de la otra.

Las armon�as son una conjunci�n de sonidos tocados al mismo tiempo, normalmente la sucesi�n de varios acordes arm�nicos est� ligado directamente con la melod�a de la canci�n.

La armon�a ha cambiado considerablemente desde la �poca de la m�sica barroca hasta la m�sica moderna, anteriormente para generar sistemas arm�nicos se usaban varios instrumentos tocando una nota en espec�fico y la conjunci�n de todos los sonidos daba como resultado la armon�a, actualmente la armon�a es creada a partir de acordes de instrumentos que puedan generar este tipo de condiciones.

Esta es una comparativa para diferenciar entre lo que ser�a una melod�a y una armon�a:

\begin{figure}[H]
	\centerline{\includegraphics[width=8cm]{melo_armo.png}}
	\caption{Melod�a y Armon�a}
	\label{fig:melo_armo}
\end{figure}

\section{Formato MIDI}

%TODO: Rf
% Deep Learning Techniques for Music Generation - {A} Survey  <- Bibliografia

MIDI (Interfaz digital de instrumentos musicales) es un est�ndar t�cnico que describe un protocolo, una interfaz digital y conectores para la interoperabilidad entre varios instrumentos musicales electr�nicos, software y dispositivos. MIDI transmite mensajes de eventos que especifican informaci�n de notas (como tono y velocidad), as� como se�ales de control para par�metros (como volumen, vibrato y se�ales de reloj). Hay cinco tipos de mensajes y aqu� solo consideramos el tipo de Canal de voz, que transmite datos de rendimiento en tiempo real a trav�s de un solo canal. Dos mensajes importantes son:

\begin{itemize}
	\item \textbf{Note on.-} Indica que una nota debe ser reproducida. Contiene la informaci�n de estado (n�mero de canal, especificado por un entero dentro de [0 15] y dos valores de datos: un n�mero de nota MIDI (el tono de la nota, un entero dentro de [0 127]) y una velocidad (que indica c�mo la nota es reproducida, un entero dentro de [0 127]). Un ejemplo es <Note on, 0, 60, 50> que interpreta como: "en el canal 1, comienza a reproducir un C medio con una velocidad de 50".
	\item \textbf{Note off.-} Indica que una nota termina. En esa situaci�n, la velocidad indica qu� tan r�pido se libera la nota. Un ejemplo es <Note off, 0, 60, 20> que interpreta como: "en el canal 1, deja de reproducir un C medio con una velocidad de 20". 
\end{itemize}

Cada evento de nota est� realmente incrustado en un fragmento de pista, una estructura de datos que contiene un valor de tiempo delta que especifica la informaci�n de temporizaci�n y el evento en s�. Un valor de tiempo delta representa la posici�n de tiempo, como un valor absoluto, del evento y podr�a representar:

\begin{itemize}
	\item \textbf{Metrical time.-} Representa el numero de pulsos desde el comienzo. Una referencia llamada divisi�n y es definida en el encabezado del archivo, especificando cuantos pulsos por nota de cuarto.
	\item \textbf{time-code-based time.-} Representa el tiempo relacionado con horas, minutos y segundos de la canci�n. 
\end{itemize}

Un ejemplo de un extracto de un archivo MIDI y su partitura correspondiente es mostrado en las siguientes figuras. La divisi�n ah sido puesta en 384 pulsos por cuarto de nota lo cual corresponde a 96 pulsos por un octavo de nota:

\begin{figure}[H]
	\centerline{\includegraphics[width=6cm]{extract_midi.png}}
	\caption{Extracto de un formato MIDI}
	\label{fig:extract_midi}
\end{figure}

\begin{figure}[H]
	\centerline{\includegraphics[width=6cm]{repres_midi.png}}
	\caption{Partitura correspondiente al extracto MIDI}
	\label{fig:repres_midi}
\end{figure}

%Rf

Dentro del formato MIDI se pueden representar en cada track un instrumento diferente eh independiente de los dem�s, esto permite una f�cil manipulaci�n de los diferentes instrumentos. A continuaci�n se muestra una tabla de todos los instrumentos que pueden ser representados en este formato:

\begin{figure}[H]
	\centerline{\includegraphics[width=12cm]{instr_midi.png}}
	\caption{Instrumentos en formato MIDI}
	\label{fig:instr_midi}
\end{figure}

\section{Algoritmo Krumhansl-Schmuckler}


\include{conclusiones}
%\include{capitulo_trabajofuturo}

% se incluye la bibliograf�a en su respectivo .bib y con el estilo deseado

%\bibliography{referencias_1}
%\bibliographystyle{IEEEtran}

\bibliography{biblio}
\bibliographystyle{IEEEtran}

%\begin{thebibliography}{1}
%\end{thebibliography}

\backmatter
\end{document}
