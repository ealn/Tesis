%IMD PNA http://na.support.keysight.com/pna/help/latest/Applications/Swept_IMD_Configure_External_Source_and_Combiner.htm
\chapter{Conclusiones}

%%-----------------------------------------------------------------------

Las diferentes arquitecturas presentadas en este documento fueron capaces de aprender y generar musica a partir de secuencias de notas de entrada. A pesar de que algunas arquitecturas no tuvieron el desempe�o esperado, los resultados finales fueron bastante alentadores.

Las redes fueron capaces de generar tanto melodias como armonias musicales en guitarra y melodias basicas de bajo.

Las secuencias de salida generadas por las redes poseen los 2 elementos basicos de la musica: sonidos y silencios. Sin embargo lo interesante de estas salidas es ver como estan interactuando estos elementos para finalmente crear una nueva pieza musical.

Al parecer las secuencias de guitarra fueron mas sencillas de aprender que las de bajo, esto es un dato bastante curioso, puesto que en bajo no se tenian tantos cambios de tonos como en la guitarra.

Despues de todas las pruebas realizadas a las arquitecturas, la que tuvo el mejor desempe�o fue sin duda la arquitectura 1, la cual cuenta con capaz intermedias de Dropout asi como tambien un optimizador RMSProp.

Es interesante analizar que aunque se use un optimizador con momento como el Nadam, este no genera un resultado sobresaliente en redes LSTM, al parecer entre mas simple sea el optimizador mejores resultados se obtienen.

Las capas de Dropout fueron vitales para el correcto aprendizaje de estas redes, ya que sin ellas, el error iba aumentando en lugar de dismunir al transcurrir las epocas. Este comportamiento lo vemos mas claramente en la arquitectura 3.

Dentro de la b�squeda por encontrar un algoritmo computacional que nos permitiera realizar una validaci�n de la salida, se encontr� con el algoritmo de Krumhansl-Schmuckler, que a pesar, de ser bastante sencillo, nos genera muy buenos valores de validaci�n.

A pesar de que las redes se entrenaron por separado para guitarra y bajo, la salida que se obtiene al conjuntar la salida de ambos instrumentos se escucha bastante bien.

El tiempo de procesamiento es algo a tomar en cuenta cuando se trabajan con muchos datos y este tipo de redes, ya que computacionalmente hablando el procesamiento de estas redes es bastante costoso, y entre mas capas se esten usando en las redes, mas tiempo tarda por cada epoca de entrenamiento.

En este trabajo �nicamente se realizaron 200 epocas de cada instrumento por cada una de las arquitecturas, ya que el tiempo promedio de procesamiento era de alrededor de 3 semanas por instrumento.

\section{Trabajos futuros}

Dentro de los trabajos futuros de esta investigaci�n seria bueno experimentar con otro tipo de arquitecturas LSTM, para ver si se logran mejores resultados a los obtenidos en este trabajo.

La infinidad de posibilidades de nuevas arquitecturas solamente se limita a la imaginaci�n del ser humano, por lo tanto no dudo que en un futuro se creen arquitecturas mas complejas para el procesamiento de secuencias. Posiblemente estas arquitecturas sean capaces de aprender y obtener buenos resultados en menor cantidad de epocas.

Seria bueno en un trabajo futuro analizar estas mismas arquitecturas usando un algoritmo de validaci�n de armon�as musicales, para ver todo el campo arm�nico y dar una mayor certeza de que se estas redes est�n generando m�sica de calidad. Este trabajo �nicamente se baso en ver que las redes fueran capaces de aprender tonalidades y generar m�sica a partir de ellas.

Se podr�an explorar la interacci�n de otros instrumentos, como por ejemplo el piano, bateria, violin, etc. para al final juntarlos y generar nuevas canciones.

La bater�a y los instrumentos de percusi�n se manejan un poco diferente a los demas instrumentos, por lo que seria interesante ver si una red es capaz de realizar ritmos bases, los cuales puedan ser integrados a nuevas canciones, as� como tambi�n ver si se generan nuevos ritmos diferentes a los que la red aprendio.

Sin duda este trabajo se podr�a extrapolar a otros estilos musicales, lo unico que se necesita es una buena base de datos para estos otros estilos musicales.
